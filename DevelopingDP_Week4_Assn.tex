% Options for packages loaded elsewhere
\PassOptionsToPackage{unicode}{hyperref}
\PassOptionsToPackage{hyphens}{url}
%
\documentclass[
  ignorenonframetext,
]{beamer}
\usepackage{pgfpages}
\setbeamertemplate{caption}[numbered]
\setbeamertemplate{caption label separator}{: }
\setbeamercolor{caption name}{fg=normal text.fg}
\beamertemplatenavigationsymbolsempty
% Prevent slide breaks in the middle of a paragraph
\widowpenalties 1 10000
\raggedbottom
\setbeamertemplate{part page}{
  \centering
  \begin{beamercolorbox}[sep=16pt,center]{part title}
    \usebeamerfont{part title}\insertpart\par
  \end{beamercolorbox}
}
\setbeamertemplate{section page}{
  \centering
  \begin{beamercolorbox}[sep=12pt,center]{part title}
    \usebeamerfont{section title}\insertsection\par
  \end{beamercolorbox}
}
\setbeamertemplate{subsection page}{
  \centering
  \begin{beamercolorbox}[sep=8pt,center]{part title}
    \usebeamerfont{subsection title}\insertsubsection\par
  \end{beamercolorbox}
}
\AtBeginPart{
  \frame{\partpage}
}
\AtBeginSection{
  \ifbibliography
  \else
    \frame{\sectionpage}
  \fi
}
\AtBeginSubsection{
  \frame{\subsectionpage}
}
\usepackage{lmodern}
\usepackage{amssymb,amsmath}
\usepackage{ifxetex,ifluatex}
\ifnum 0\ifxetex 1\fi\ifluatex 1\fi=0 % if pdftex
  \usepackage[T1]{fontenc}
  \usepackage[utf8]{inputenc}
  \usepackage{textcomp} % provide euro and other symbols
\else % if luatex or xetex
  \usepackage{unicode-math}
  \defaultfontfeatures{Scale=MatchLowercase}
  \defaultfontfeatures[\rmfamily]{Ligatures=TeX,Scale=1}
\fi
% Use upquote if available, for straight quotes in verbatim environments
\IfFileExists{upquote.sty}{\usepackage{upquote}}{}
\IfFileExists{microtype.sty}{% use microtype if available
  \usepackage[]{microtype}
  \UseMicrotypeSet[protrusion]{basicmath} % disable protrusion for tt fonts
}{}
\makeatletter
\@ifundefined{KOMAClassName}{% if non-KOMA class
  \IfFileExists{parskip.sty}{%
    \usepackage{parskip}
  }{% else
    \setlength{\parindent}{0pt}
    \setlength{\parskip}{6pt plus 2pt minus 1pt}}
}{% if KOMA class
  \KOMAoptions{parskip=half}}
\makeatother
\usepackage{xcolor}
\IfFileExists{xurl.sty}{\usepackage{xurl}}{} % add URL line breaks if available
\IfFileExists{bookmark.sty}{\usepackage{bookmark}}{\usepackage{hyperref}}
\hypersetup{
  pdftitle={Week4\_Assignment},
  pdfauthor={KS},
  hidelinks,
  pdfcreator={LaTeX via pandoc}}
\urlstyle{same} % disable monospaced font for URLs
\newif\ifbibliography
\usepackage{color}
\usepackage{fancyvrb}
\newcommand{\VerbBar}{|}
\newcommand{\VERB}{\Verb[commandchars=\\\{\}]}
\DefineVerbatimEnvironment{Highlighting}{Verbatim}{commandchars=\\\{\}}
% Add ',fontsize=\small' for more characters per line
\usepackage{framed}
\definecolor{shadecolor}{RGB}{248,248,248}
\newenvironment{Shaded}{\begin{snugshade}}{\end{snugshade}}
\newcommand{\AlertTok}[1]{\textcolor[rgb]{0.94,0.16,0.16}{#1}}
\newcommand{\AnnotationTok}[1]{\textcolor[rgb]{0.56,0.35,0.01}{\textbf{\textit{#1}}}}
\newcommand{\AttributeTok}[1]{\textcolor[rgb]{0.77,0.63,0.00}{#1}}
\newcommand{\BaseNTok}[1]{\textcolor[rgb]{0.00,0.00,0.81}{#1}}
\newcommand{\BuiltInTok}[1]{#1}
\newcommand{\CharTok}[1]{\textcolor[rgb]{0.31,0.60,0.02}{#1}}
\newcommand{\CommentTok}[1]{\textcolor[rgb]{0.56,0.35,0.01}{\textit{#1}}}
\newcommand{\CommentVarTok}[1]{\textcolor[rgb]{0.56,0.35,0.01}{\textbf{\textit{#1}}}}
\newcommand{\ConstantTok}[1]{\textcolor[rgb]{0.00,0.00,0.00}{#1}}
\newcommand{\ControlFlowTok}[1]{\textcolor[rgb]{0.13,0.29,0.53}{\textbf{#1}}}
\newcommand{\DataTypeTok}[1]{\textcolor[rgb]{0.13,0.29,0.53}{#1}}
\newcommand{\DecValTok}[1]{\textcolor[rgb]{0.00,0.00,0.81}{#1}}
\newcommand{\DocumentationTok}[1]{\textcolor[rgb]{0.56,0.35,0.01}{\textbf{\textit{#1}}}}
\newcommand{\ErrorTok}[1]{\textcolor[rgb]{0.64,0.00,0.00}{\textbf{#1}}}
\newcommand{\ExtensionTok}[1]{#1}
\newcommand{\FloatTok}[1]{\textcolor[rgb]{0.00,0.00,0.81}{#1}}
\newcommand{\FunctionTok}[1]{\textcolor[rgb]{0.00,0.00,0.00}{#1}}
\newcommand{\ImportTok}[1]{#1}
\newcommand{\InformationTok}[1]{\textcolor[rgb]{0.56,0.35,0.01}{\textbf{\textit{#1}}}}
\newcommand{\KeywordTok}[1]{\textcolor[rgb]{0.13,0.29,0.53}{\textbf{#1}}}
\newcommand{\NormalTok}[1]{#1}
\newcommand{\OperatorTok}[1]{\textcolor[rgb]{0.81,0.36,0.00}{\textbf{#1}}}
\newcommand{\OtherTok}[1]{\textcolor[rgb]{0.56,0.35,0.01}{#1}}
\newcommand{\PreprocessorTok}[1]{\textcolor[rgb]{0.56,0.35,0.01}{\textit{#1}}}
\newcommand{\RegionMarkerTok}[1]{#1}
\newcommand{\SpecialCharTok}[1]{\textcolor[rgb]{0.00,0.00,0.00}{#1}}
\newcommand{\SpecialStringTok}[1]{\textcolor[rgb]{0.31,0.60,0.02}{#1}}
\newcommand{\StringTok}[1]{\textcolor[rgb]{0.31,0.60,0.02}{#1}}
\newcommand{\VariableTok}[1]{\textcolor[rgb]{0.00,0.00,0.00}{#1}}
\newcommand{\VerbatimStringTok}[1]{\textcolor[rgb]{0.31,0.60,0.02}{#1}}
\newcommand{\WarningTok}[1]{\textcolor[rgb]{0.56,0.35,0.01}{\textbf{\textit{#1}}}}
\usepackage{graphicx,grffile}
\makeatletter
\def\maxwidth{\ifdim\Gin@nat@width>\linewidth\linewidth\else\Gin@nat@width\fi}
\def\maxheight{\ifdim\Gin@nat@height>\textheight\textheight\else\Gin@nat@height\fi}
\makeatother
% Scale images if necessary, so that they will not overflow the page
% margins by default, and it is still possible to overwrite the defaults
% using explicit options in \includegraphics[width, height, ...]{}
\setkeys{Gin}{width=\maxwidth,height=\maxheight,keepaspectratio}
% Set default figure placement to htbp
\makeatletter
\def\fps@figure{htbp}
\makeatother
\setlength{\emergencystretch}{3em} % prevent overfull lines
\providecommand{\tightlist}{%
  \setlength{\itemsep}{0pt}\setlength{\parskip}{0pt}}
\setcounter{secnumdepth}{-\maxdimen} % remove section numbering

\title{Week4\_Assignment}
\author{KS}
\date{5/3/2021}

\begin{document}
\frame{\titlepage}

\begin{frame}{Introduction}
\protect\hypertarget{introduction}{}

This data set contains statistics, in arrests per 100,000 residents for
assault, murder, and rape in each of the 50 US states in 1973. Also
given is the percent of the population living in urban areas. A data
frame with 50 observations on 4 variables.

{[},1{]} Murder numeric Murder arrests (per 100,000) {[},2{]} Assault
numeric Assault arrests (per 100,000) {[},3{]} UrbanPop numeric Percent
urban population {[},4{]} Rape numeric Rape arrests (per 100,000)

\end{frame}

\begin{frame}{Shiny Web Application}
\protect\hypertarget{shiny-web-application}{}

\begin{itemize}
\tightlist
\item
  The Shiny Web Application developed by using USArrests Dataset from
  inbuilt R package.
\item
  The developed shiny App has 2 parts, a drop down box on left panel and
  main panel is on the right
\item
  The Crime drop down box facilitates the user to choose 3 different
  crime rates in USA 1973
\item
  Based on the selection in the drop down menu, it is possible to see
  the associated data,map,regression diagnostics on the tabs in the main
  panel.
\end{itemize}

Screenshots of the Server.R file

\begin{figure}
\centering
\includegraphics{/Users/kavis/Pictures/Saved Pictures/ServerR.jpg}
\caption{Alt text}
\end{figure}

\end{frame}

\begin{frame}[fragile]{Slide with R Output}
\protect\hypertarget{slide-with-r-output}{}

\begin{Shaded}
\begin{Highlighting}[]
\KeywordTok{summary}\NormalTok{(USArrests)}
\end{Highlighting}
\end{Shaded}

\begin{verbatim}
##      Murder          Assault         UrbanPop          Rape      
##  Min.   : 0.800   Min.   : 45.0   Min.   :32.00   Min.   : 7.30  
##  1st Qu.: 4.075   1st Qu.:109.0   1st Qu.:54.50   1st Qu.:15.07  
##  Median : 7.250   Median :159.0   Median :66.00   Median :20.10  
##  Mean   : 7.788   Mean   :170.8   Mean   :65.54   Mean   :21.23  
##  3rd Qu.:11.250   3rd Qu.:249.0   3rd Qu.:77.75   3rd Qu.:26.18  
##  Max.   :17.400   Max.   :337.0   Max.   :91.00   Max.   :46.00
\end{verbatim}

\begin{Shaded}
\begin{Highlighting}[]
\KeywordTok{require}\NormalTok{(graphics)}
\KeywordTok{pairs}\NormalTok{(USArrests, }\DataTypeTok{panel =}\NormalTok{ panel.smooth, }\DataTypeTok{main =} \StringTok{"USArrests data"}\NormalTok{)}
\end{Highlighting}
\end{Shaded}

\includegraphics{DevelopingDP_Week4_Assn_files/figure-beamer/cars-1.pdf}

\begin{Shaded}
\begin{Highlighting}[]
\CommentTok{## Difference between 'USArrests' and its correction}
\NormalTok{USArrests[}\StringTok{"Maryland"}\NormalTok{, }\StringTok{"UrbanPop"}\NormalTok{] }\CommentTok{# 67 -- the transcription error}
\end{Highlighting}
\end{Shaded}

\begin{verbatim}
## [1] 67
\end{verbatim}

\begin{Shaded}
\begin{Highlighting}[]
\NormalTok{UA.C <-}\StringTok{ }\NormalTok{USArrests}
\NormalTok{UA.C[}\StringTok{"Maryland"}\NormalTok{, }\StringTok{"UrbanPop"}\NormalTok{] <-}\StringTok{ }\FloatTok{76.6}

\CommentTok{## also +/- 0.5 to restore the original  <n>.5  percentages}
\NormalTok{s5u <-}\StringTok{ }\KeywordTok{c}\NormalTok{(}\StringTok{"Colorado"}\NormalTok{, }\StringTok{"Florida"}\NormalTok{, }\StringTok{"Mississippi"}\NormalTok{, }\StringTok{"Wyoming"}\NormalTok{)}
\NormalTok{s5d <-}\StringTok{ }\KeywordTok{c}\NormalTok{(}\StringTok{"Nebraska"}\NormalTok{, }\StringTok{"Pennsylvania"}\NormalTok{)}
\NormalTok{UA.C[s5u, }\StringTok{"UrbanPop"}\NormalTok{] <-}\StringTok{ }\NormalTok{UA.C[s5u, }\StringTok{"UrbanPop"}\NormalTok{] }\OperatorTok{+}\StringTok{ }\FloatTok{0.5}
\NormalTok{UA.C[s5d, }\StringTok{"UrbanPop"}\NormalTok{] <-}\StringTok{ }\NormalTok{UA.C[s5d, }\StringTok{"UrbanPop"}\NormalTok{] }\OperatorTok{-}\StringTok{ }\FloatTok{0.5}
\end{Highlighting}
\end{Shaded}

\end{frame}

\begin{frame}{Links}
\protect\hypertarget{links}{}

Follow the links for * Shiny app viewer in Rpubs.

\begin{itemize}
\tightlist
\item
  The code can be found at
\end{itemize}

\end{frame}

\end{document}
